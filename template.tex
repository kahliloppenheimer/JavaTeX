%@
\renewcommand{\studentName}{@
%@
@}

	\begin{center}
		\Huge Problem Set \assignmentNum\\
		\huge \studentName
	
	\large 
	\section*{Raw Score}
	\texttt{
			\begin{tabular}{llll}
			 	Problem&Score&Average&Maximum\\\hline\hline
			 	@&@&4.48&5\\
			 	@&@&4.49&5\\
			 	@&@&4.45&5\\
			 	@&@&4.48&5\\
			 	@&@&4.41&5\\
			 	@&@&4.45&5\\
			 	@&@&4.40&5\\
			 	@&@&4.42&5\\
			 	@&@&4.37&5\\
			 	@&@&4.32&5\\
			 	@&@&4.17&5\\
			 	@&@&3.86&5\\
			 	@&@&4.18&5\\
			 	@&@&2.80&5\\
			 	@&@&2.85&5\\
			 	@&@&3.24&5\\
			 	@&@&3.55&5\\
			 	\hline
			 	@&@&68.98&85\\\hline
			\end{tabular}}

	\section*{Adjusted Score}
	\texttt{
			\begin{tabular}{llll}
			 	Problem&Score&Average&Maximum\\\hline\hline
			 	@&@&8.86&10\\
			 	@&@&7.51&10\\
			 	@&@&6.09&10\\
			 	@&@&7.05&10\\
			 	\hline
			 	@&@&29.52&40\\\hline
			\end{tabular}}
  %second table

	\section*{Your Grade: 
	%@
	@ 
	\%}

		\section*{Average Grade: 73.82\%}

\newpage

\section*{Rubric}
\end{center}
\tiny
\begin{tabularx}{\linewidth}{XXXXXXX}
   & 0          & 1                                                                                                                               & 2                                                                                                                                                                               & 3                                                                                                                                                                                                                                                                                                                                                                                                                                                                                                                  & 4                                                                                                                                       & 5                                                                                          \\ \hline
1  & No answer. & Solution consists of something correct (i.e. the definition of big-O), but nothing substantially relevant to the actual problem & Solution is partial or mostly incorrect (either due to numerous small errors or to larger logic errors)                                                                         & Solution is somewhere in between correct and incorrect but demonstrates some understanding of subject material                                                                                                                                                                                                                                                                                                                                                                                                     & Solution is almost completely correct with few possible small errors (arithmetic, improper proof logic, etc.)                           & Solution is completely correct and demonstrates thorough understanding of subject material \\ \hline
2  & No answer. & Solution consists of something correct (i.e. the definition of big-O), but nothing substantially relevant to the actual problem & Solution is partial or mostly incorrect (either due to numerous small errors, larger logic errors, or incorrect interpretation of the problem)                                  & Solution is somewhere in between correct and incorrect but demonstrates some understanding of subject material. For problem 4, this includes supplying functions for which f(n) \textless h(n) for infinite n but for which f(n) is still big-omega of h(n) (i.e. f(n) = 1 and h(n) = 5). Notably also for problem 4, any solution that invokes a cyclical function (i.e. sin(x), (-1)\textasciicircum n, etc.) will not work since f(n) and h(n) will still be only a constant coefficient away from one another. & Solution is almost completely correct with few possible small errors (arithmetic, improper proof logic, etc.)                           & Solution is completely correct and demonstrates thorough understanding of subject material \\ \hline
3a & No answer. & Does not answer the question                                                                                                    & Solution is partial or mostly incorrect (either due to numerous small errors or to larger logic errors) or provides only an example of this rather than any display of pattern. & Solution is somewhere in between correct and incorrect but demonstrates some understanding of subject material                                                                                                                                                                                                                                                                                                                                                                                                     & Solution is almost completely correct with few possible small errors (arithmetic, improper proof logic, etc.)                           & A really strong argument showing why this occurs.                                          \\ \hline
3b & No answer. & Invalid code                                                                                                                    & Uses log library, does some bizzare computation.                                                                                                                                & Uses exponentiation                                                                                                                                                                                                                                                                                                                                                                                                                                                                                                & Writes two loops instead of method call, or is correct but contains a gross amount of Java, or is too vague of a loop or call to be a 5 & Solution is completely correct and demonstrates thorough understanding of subject material \\ \hline
4  & No answer. & Some algorithm, does not address the problem at hand.                                                                           & Solution is partial or mostly incorrect (either due to numerous small errors or to larger logic errors) or simply cannot work.                                                  & Solution does not account for side cases, and it is not clear from code comments whether the value exists in the array. Infinite loop. Does not return anything.                                                                                                                                                                                                                                                                                                                                                   & Solution fails to account for some side cases, like the number not being in the array or being in either of the two partitions.         & Proper pseudo code and accounts for most if not all side cases.                            \\ \hline                       
\end{tabularx}

\newpage